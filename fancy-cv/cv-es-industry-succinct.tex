%!TEX TS-program = xelatex
\documentclass[]{friggeri-cv}
\usepackage{afterpage}
\usepackage{hyperref}
\usepackage{color}
\usepackage{xcolor}
\hypersetup{
    pdftitle={},
    pdfauthor={},
    pdfsubject={},
    pdfkeywords={},
    colorlinks=false,       % no lik border color
   allbordercolors=white    % white border color for all
}
\RequirePackage{xcolor}
\definecolor{pblue}{HTML}{0395DE}

\usepackage{enumitem}

\begin{document}
\header{Victor M.}{Mendiola-Lau}{Ingeniero de Software y Consultor}
      
% Fake text to add separator      
\fcolorbox{white}{gray}{\parbox{\dimexpr\textwidth-2\fboxsep-2\fboxrule}{%
.....
}}

% In the aside, each new line forces a line break
\begin{aside}
  \section{Dirección}
  	~
    Miami, FL, USA
    (+1) 786-630-0166
    \href{mailto:ryuzakyl@gmail.com}{\textbf{ryuzakyl@}gmail.com}
    ~
    ~
    ~
  \section{Perfiles online}
  	~
    \href{https://ryuzakyl.github.io}{{\scriptsize ryuzakyl.github.io}}
    \href{https://github.com/ryuzakyl}{{\scriptsize github.com/ryuzakyl}}
    \href{https://www.linkedin.com/in/victormendiolalau}{{\scriptsize linkedin.com/in/VictorMendiolaLau}}
%	\href{https://www.researchgate.net/profile/Victor_Mendiola-Lau}{{\scriptsize researchgate.net/Victor\char`_Mendiola-Lau}}
    ~
    ~
    ~
  \section{Cualidades}
    ~
	\smartdiagram[bubble diagram]{
        \textbf{Curiosity},
        \textbf{Team Work},
        \textbf{Initiative},
        \textbf{Responsibility},
        \textbf{Dedicated}        
        
    }
    ~
    ~
    ~
  \section{Sobre mí}
    ~
    Soy un eterno estudiante, con gusto por la creatividad, tanto en la programación como en la cocina. Aventurero de corazón, exploro el mundo y las situaciones desafiantes con una mente abierta.  Me encanta debatir ideas y compartir con amigos y familiares por todo el mundo.
 \end{aside}

\section{Experiencia Laboral}
\begin{entrylist}
  \entry
    {2022 - 2023}
    {Ingeniero de DevOps}
    {Rolla Finance, USA}
    {Mis principales responsabilidades incluyen la definición e implementación de nuestra infraestructura Cloud, Kubernetes, Monitoreo y Seguridad.\\}

  \entry
    {2020 - 2021}
    {Interno de Ciberseguridad}
    {EY, Málaga, España}
    {Como parte del equipo de Pentesting Team, trabajé en la generación de informes de conformidad de seguridad con las guías CIS Benchmarks.\\}

  \entry
    {}
    {Desarrollador Web Fullstack Senior}
    {Novoshore, Málaga, España}
    {Amplia gama de responsabilidades en un cliente de Fintech Startup Suizo. Tecnologías de Frontend: Angular. Tecnologías de Backend: NodeJS, Spring Boot, Java, PostgreSQL.\\}

  \entry
    {2019 - 2020}
    {Desarrollador Web Fullstack}
    {Ottawa, Canadá} % {Iversoft - Digital consultancy & technology partner. (https://www.iversoft.ca/)}
    {Construcción del sitio web de una organización. Tecnologías: Wordpress, PHP, HTML, CSS3, JavaScript, MySQL, etc.\\}

  \entry
    {2018 - 2019}
    {Ingeniero de Software}
    {Kitsune Technologies, Tennessee, USA}
    {Amplia gama de responsabilidades, incluyendo Análisis de Datos, Visión por Computadora, Desarrollo Web, implementación de servicios, etc.\\}

  \entry
    {}
    {Desarrollador de Backend en Python/Flask}
    %{Ezequiel}
    {} 
    {Selenium, ChromeDriver y Celery/Redis para desarrollar una API REST que simula el comportamiento del usuario para eludir CAPTCHAs.\\}

  \entry
    {2017 - 2018}
    {Desarrollador Web Fullstack}
    % {HRSG - Human Resource Systems Group Ltd. (https://www.hrsg.ca/)}
    {}
    {Aplicaciones web como soluciones al manejo de las fuerzas de trabajo. Tecnologías: PHP, HTML, CSS3, JavaScript, jQuery, MySQL, etc.\\}

  \entry
    {}
    {Desarrollador de Backend en Python/Django}
    {SBY Technologies, Florida, USA}
    {Desarrollo de un API desde cero con autenticación mediante JWT, generación automática de reportes (.pdf, .csv), etc. También desempeñé el rol de ingeniero de DevOps y Administrador de Sistemas en Linux.\\}
    
  \entry
    {2016}
    {Desarrollador de Backend en Ruby/Rails}
    {Ksabes, La Habana, Cuba}
    {Diseño e implementación propia de un API basada en Json Web Token (JWT) para la autenticación.\\}

  \entry
    {2014 - 2015}
    {Desarrollador de Ruby}
    {IRStrat, México D.F., México}
    {Diseño e implementación propia de un servicio concurrente y multi-hilos para consumir información de la bolsa de valores en tiempo real.\\}

  \entry
    {2013 - 2017}
    {Ingeniero de Data Science y Machine Learning}
    {CENATAV, La Habana, Cuba}
    {Utilicé Data Science y Machine Learning para diseñar e implementar sistemas de Reconocimiento de Patrones en C\#/C++. Además trabajé como Ingeniero de Software diseñando e implementando algoritmos eficientes.\\}
\end{entrylist}

\section{Educación/Formación académica}
\begin{entrylist}
  \entry
    {2019 - 2021}
    {Máster en Seguridad Informática}
    {Universidad de Jaén, España}
    {
      Materias principales: Aplicaciones Seguras en la Nube, Seguridad en las Redes, Detección de Intrusiones, Ingeniería Inversa, Análisis de Malware, Criptografía Avanzada.\\
      Tesis: \emph{``Estudio e implantación de medidas de seguridad para clústeres de Kubernetes''.}\\
    }

  \entry
    {2008 - 2013}
    {B.Sc. en Ciencia de la Computación}
    {Universidad de La Habana, Cuba}
    {Título de Oro (5.13 de 5.0) en Ciencia de la Computación.\\ Materias principales: Programación, Estructuras de Datos y Algoritmos, Teoría de la Complejidad Computacional, Matemática, Investigación Operacional, Matemática Numérica e Inteligencia Artificial.\\
    Tesis: \emph{``Reconocimiento de iris utilizando Análisis de Datos Funcionales''.}}

%  \entry
%    {2004 - 2007}
%    {Bachelor Diploma}
%    {IPVCE Vladimir Ilich Lenin, Havana, Cuba}
%    {Bachelor diploma with focus on these subjects: Matematics and Computer Science.}
\end{entrylist}

\section{Idiomas}
\begin{entrylist}
  \entry
    {\textbf{Español}}
    {}
    {}
    {Lengua materna}

  \entry
    {\textbf{Inglés}}
    {}
    {}
    {
    \textbf{Usuario experto}: Puntuación del IELTS de 7.0 (nivel \textbf{C1} en el CEFR).% \\    
      % Graduado \textbf{\emph{summa cum laude}} de la Escuela de Idiomas \emph{Abraham Lincoln}.
    }

  \entry
    {\textbf{Francés}}
    {}
    {}
    {Diploma de estudios de la Lengua Francesa (\textbf{DELF B2}).}
\end{entrylist}

% \pagebreak

\section{Investigación \& Docencia}
\begin{entrylist}
  \entry
    {2017 - 2018}
    {Investigador Visitante}
    {Universidad de Sevilla, España}
    {Investigador en temas de Aprendizaje de Máquina y Análisis de Datos.\\}

  \entry
    {}
    {Profesor Instructor}
    {Universidad de La Habana, Cuba}
    {Instructor de Matemática en temáticas como Teoría de Conjuntos, Funciones, Álgebra, Cálculo, Matemática Numérica y Ecuaciones Diferenciales Ordinarias.\\}
\end{entrylist}

\begin{entrylist}
  \entry
    {2013 - 2017}
    {Investigador}
    {CENATAV, La Habana, Cuba}
    {Diseño y desarrollo de Sistemas de Reconocimiento de Patrones. Investigador junior y estudiante de doctorado. Con intereses en temas como Aprendizaje de Máquina, Análisis de Datos, Representación de Datos y Visión por Computadoras. Tecnologías: Python, Numpy, SciPy, Matplotlib, scikit-learn, etc.\\}

  \entry
    {2012}
    {Pasantía de investigación}
    {CENATAV, La Habana, Cuba}
    {Diseño e implementación de un método para codificar el iris basado en el Análisis de Datos Funcionales (FDA).\\}

  \entry
    {2010 - 2013}
    {Instructor asistente}
    {Universidad de La Habana, Cuba}
    {Instructor  de Programación y Algoritmos para la carrera de Matemática en la Facultad de Matemática y Computación (MATCOM).\\}
\end{entrylist}

% \section{Tecnologías}
% \begin{entrylist}
%   \entry
%     {\textbf{Lenguajes}}
%     {}
%     {}
%     {\textbf{C\#} (+9 años), \textbf{Python} (+7 años), \textbf{C/C++} (+5 años), \textbf{Ruby} (+2 años), \textbf{PHP} (+2 años). Otros lenguajes incluyen \textbf{Go} y \textbf{Java}.}

%   \entry
%     {\textbf{IDEs\&Editores}}
%     {}
%     {}
%     {\textbf{Visual Studio}, \textbf{VS Code} e \textbf{IDEs de JetBrains}.}

%   \entry
%     {\textbf{ML \& DA}}
%     {}
%     {}
%     {Tecnologías basadas en Python para el Aprendizaje de Máquina y el Análisis de Datos: \textbf{Numpy}, \textbf{SciPy}, \textbf{Pandas}, \textbf{scikit-learn} and \textbf{Matplotlib}. Además del lenguaje de programación y ambiente \textbf{MATLAB}.}

%   \entry
%     {\textbf{Web}}
%     {}
%     {}
%     {\textbf{HTML}, \textbf{CSS3}, \textbf{JavaScript}, \textbf{Vue.js}, \textbf{jQuery}, \textbf{PHP}, \textbf{Django}, \textbf{Django Rest Framework}, \textbf{ASP.NET MVC/Core} y \textbf{Ruby on Rails}.}

%   \entry
%     {\textbf{Cloud}}
%     {}
%     {}
%     {\textbf{AWS} y \textbf{Google Cloud Platform}.}

%   \entry
%     {\textbf{Desktop}}
%     {}
%     {}
%     {\textbf{.NET}, \textbf{.NET Core} y \textbf{Qt}.}

% \end{entrylist}

% \begin{entrylist}

%   \entry
%     {\textbf{DVCS}}
%     {}
%     {}
%     {\textbf{Git} (+5 años) (\textbf{GitHub}, \textbf{GitLab} y \textbf{Bitbucket}).}
    
%   \entry
%     {\textbf{DBMS}}
%     {}
%     {}
%     {\textbf{PostgreSQL}, \textbf{Microsoft SQL Server}, \textbf{MySQL} y \textbf{SQLite}.}
    
%   \entry
%     {\textbf{CV}}
%     {}
%     {}
%     {\textbf{OpenCV}, \textbf{EmguCV} y \textbf{OpenCV-Python bindings}.}    

%   \entry
%     {\textbf{SO}}
%     {}
%     {}
%     {\textbf{GNU/Linux} (+4 años) (\textbf{Ubuntu}, \textbf{Linux Mint}) y \textbf{Microsoft Windows} (+12 años).}
% \end{entrylist}

% \section{Otros Logros}
% \begin{itemize}[noitemsep, nolistsep]

% 	\item Obtuvo un promedio de 5.13 sobre una escala de 5.0 puntos\footnote{Los puntos adicionales fueron producto de resultados académicos relevantes.}.\\

% 	\item Concursante de la Final Cubana de la ACM-ICPC (ediciones del 2011 y 2013)\footnote{Competir por un puesto en el concurso de la Final Caribeña de la ACM-ICPC.}.\\

% 	\item Ganador del $1^{er}$ premio en la Jornada Científica Estudiantil de la Facultad de Matemática y Computación del curso 2010-2011.\\
	
% 	\item Ha realizado contribuciones al artículo de Wikipedia ``\emph{Teoría de la complejidad computacional}".\\
		
% 	\item Ganador de la medalla de plata en el concurso provincial de matemática en $12^{mo}$ grado de La Habana.\\
	
% 	% \item From 2004 until 2007 was selected as part of advanced groups in Mathematics.\\	
	
% 	% \item Best candidate\footnote{Highest grade.} of the \emph{Plaza de la Revolución} locality at the IPVCE Vladimir Ilich Lenin admission tests in 2003-2004.\\	
	
% 	% \item Valedictorian from my elementary school in 2001.\\	
	
% \end{itemize}

\end{document}
